\documentclass{article}
\usepackage[utf8]{inputenc}
\usepackage{amsmath, amssymb, amsthm}
\usepackage{tikz}
\usepackage{wrapfig}
\usepackage[margin=15mm]{geometry}


\newcommand{\R}{\mathbb{R}}
\newcommand{\Z}{\mathbb{Z}}
\newcommand{\N}{\mathbb{N}}
\newcommand{\Q}{\mathbb{Q}}
\newcommand{\diag}{\text{diag}}
\newcommand{\rad}{\text{rad\ }}
\newcommand{\e}{\epsilon}

\newtheorem*{lemma}{Lemma}

\begin{document}
\begin{center}
\subsection*{Abbot Chapter 1 Section 2}
\end{center}

Heman Gandhi
\hfill
2024-10-22\\

\subsubsection*{Exercise 1.2.1-a}

\textit{This follows the $exercise\_1\_2\_1$ section in Lean, so might be overly detailed. I'm also pretentiously going to use lemmas that correspond to the Lean theorems.}

\begin{lemma}
First, we show that for $n \in \N$, we know that there is an $x \in \N$ so that $3x = n$,
$3x + 1 = n$, or $3x + 2 = n$.
\end{lemma}

\begin{proof}
We proceed by induction, showing that $3 \cdot 0 = 0$, then
if the statement holds for $n$, we proceed by cases and show that for the same $x$ we reached for
$n$, we either have $3x + 1 = n + 1$, $3x + 2 = n + 1$, or $3(x + 1) = n + 1$.
\end{proof}

\begin{lemma}
If $n \in \N$, then $3 \nmid n$ if and only if for some $x \in \N$, $3x + 1 = n$
or $3x + 2 = n$.
\end{lemma}

\begin{proof}
We can prove the forward direction by the above statement: since $3x = n$ would contradict
the assumption that $3 \nmid n$.

The reverse direction is simplest by contradiction: if we have the $x$ with
remainder 1 or 2, we cannot find some $y$ so that $3y = n$, since we'd form the equation $3(y - x) = r$ for $r$
being 1 or 2, which is absurd since 3 cannot divide a non-zero number less than itself.
\end{proof}

For the final lemma: we show that

\begin{lemma}
For $a, b \in \N$ if $3 \mid ab$, $3 \mid a$ or $3 \mid b$.
\end{lemma}

\begin{proof}
We show the contrapositive: assuming $3 \nmid a$ and $3 \nmid b$, we have $3x_a + r_a = a$ and
$3x_b + r_b = b$ for $x_a, x_b \in \N$ and $r_a, r_b \in \{1, 2\}$ from the forward direction of the above.
We compute $ab$ with the above in all four cases:
\begin{enumerate}
    \item if $r_a, r_b = 1$, then $ab = 9 x_a x_b + 3 x_a + 3 x_b + 1 = 3(3 x_a x_b + x_a + x_b) + 1$;
    \item if $r_a = 1, r_b = 2$, then $ab = 9 x_a x_b + 6 x_a + 3 x_b + 1 = 3(3 x_a x_b + 2 x_a + x_b) + 2$;
    \item if $r_a = 2, r_b = 1$, then $ab = 9 x_a x_b + 3 x_a + 6 x_b + 1 = 3(3 x_a x_b + x_a + 2 x_b) + 2$;
    \item if $r_a, r_b = 2$, then $ab = 9 x_a x_b + 6 x_a + 6 x_b + 4 = 3(3 x_a x_b + 2 x_a + 2 x_b + 1) + 1$.
\end{enumerate}
In all the cases, we can express $ab = 3y + r$ for $y \in \N$ and $r \in \{1, 2\}$ and apply the backwards direction of
the lemma above to conclude $3 \nmid ab$, showing the contrapositive.
\end{proof}

\begin{lemma}
$\sqrt{3}$ is irrational.
\end{lemma}

\begin{proof}
For contradiction, let $a, b \in \N$ and ${a^2 \over b^2} = 3$. Without loss of generality, we can assume that $a$ and $b$ don't
share factors. Rewriting this as $a^2 = 3 b^2$, we see that $3 \mid a^2$, so the above gives us that $3 \mid a$.
Hence, we write $a = 3d$, so $a^2 = 9 d^2 = 3 b^2$, which means that $b^2 = 3 d^2$, so $3 \mid b$. This contradicts the assumption
that $a$ and $b$ don't share factors.
\end{proof}

The proof would more-or-less work for $\sqrt{6}$.

\textit{The proof below is kept brief, emphasizing the differences between the case for $\sqrt{3}$ and $\sqrt{6}$. Lean has all the details.}

\begin{lemma}
$\sqrt{6}$ is irrational.
\end{lemma}

\begin{proof}
We follow the above structure of the proof for $\sqrt{3}$: we have that for $n \in \N$,
there is an $x \in \N$ and $0 \leq r \leq 5$ with $r \in \N$. Futhermore, $r = 0$ if and only if $6 \mid n$.
Finally, instead of the prior lemma, we show $6 \mid a^2$ implies $6 \mid a$.
If we attempt the contrapositive, we have 5 cases for $a = 6x + r$, where we rewrite $a^2 = 6 (6x^2 + 2rx) + r^2 = 6y + r^2$:
\begin{enumerate}
    \item if $r = 1$, $r^2 = 1$, so $a^2 = 6y + 1$;
    \item if $r = 2$, $r^2 = 4$, so $a^2 = 6y + 4$;
    \item if $r = 3$, $r^2 = 9$, so $a^2 = 6(y + 1) + 3$;
    \item if $r = 4$, $r^2 = 16$, so $a^2 = 6(y + 2) + 4$;
    \item if $r = 5$, $r^2 = 25$, so $a^2 = 6(y + 4) + 1$.
\end{enumerate}
This shows that if $a \nmid 6$, $a^2 \nmid 6$. We can proceed with this, showing that if $a^2 = 6b^2$ and
$a$ and $b$ don't share factors, we get $6 \mid a$ and $6 \mid b$.
\end{proof}

\subsubsection*{Exercise 1.2.1-b}

The proof of theorem 1.1.1 breaks down at the first supposition that $p^2 = 4q^2$ as $p = 2, q = 1$ already suffices.
Furthermore, $4 \mid 6^2$ but $4 \nmid 6$, so in the next step, we cannot conclude anything from the fact that $4 \mid p^2$.

\subsubsection*{Exercise 1.2.2}

\begin{proof}
If we suppose that $2^r = 3$ for some rational $r$, we can show that $2^p = 3^q$ for some $p, q \in \N$ with $p > 1$.
However, we can show that all such powers of 2 are even whereas all powers of 3 are odd. Therefore, such a $p, q$
cannot exist.
\end{proof}

\subsubsection*{Exercise 1.2.3-a}

We use $A_n = \{ 2 ^ {(n + k)} : k \in \N \}$.
The $\cap_{n=1}^{\infty} A_n = \emptyset$.

\subsubsection*{Exercise 1.2.3-b}

\textit{I couldn't find the well-ordering principle in the text, but I think that's clearly the way to prove this.}

Since all $A_n$ are finite, and each $A_n$ is a subset of the prior ones, there must be one of a minimal cardinality $c$,
and $c \geq 1$ since even the smallest set must be nonempty.
All sets will contain this smallest set, so it will be the intersection, which is, therefore, finite and nonempty.

\subsubsection*{Exercise 1.2.3-c}

Take $A = \{1, 2, 3\}, B = \{2, 4\},$ and $C = \{3, 6\}$ to observe that $A \cap (B \cup C) = \{2, 3\} \neq (A \cap B) \cup C = \{2, 3, 6\}$.

\subsubsection*{Exercise 1.2.3-d}

Yes: if $x \in A \cap (B \cap C)$, $x$ is in all of $A, B,$ and $C$, so $x \in (A \cap B) \cap C$ and similarly in the reverse direction.

\subsubsection*{Exercise 1.2.3-e}

Yes: if $x \in A \cap (B \cup C)$, we know $x \in A$ and $x \in B$ or $x \in C$, so either $x \in A \cap B$ since $x \in B$, or $x \in A \cap C$ otherwise.
In reverse, if $x \in A \cap B$, in $x \in A$ and $B$, so $x \in A \cap (B \cup C)$ and similarly if $x \in C$.

\subsubsection*{Exercise 1.2.4}

Let $A_i$ be the subset of $\N$ formed by numbers with $i$ prime factors (forgetting exponents -- so $12 \in A_2$).
Then, by unique factorization, each $n \in \N$ is in exactly one $A_i$. Hence, if $i \neq j$, $A_i \cap A_j = \emptyset$,
and $\cup_{i=1}^{\infty} A_i = \N$.

\subsubsection*{Exercise 1.2.5-a}

\begin{proof}
Let $x \in (A \cap B)^c$. Then, $x \not\in A \cap B$, so it is not the case that $x \in A$ and $x \in B$; therefore, by the propositional DeMorgan's rule,
$x \not\in A$ or $x \not\in B$, so $x \in A^c \cup B^c$.
\end{proof}

\subsubsection*{Exercise 1.2.5-b}

\begin{proof}
Put $x \in A^c \cup B^c$. Then, $x \not\in A$ or $x \not\in B$, which, by the propositional DeMorgan's rule, gives $x \not\in (A \cap B)^c$.
\end{proof}

\subsubsection*{Exercise 1.2.5-c}

\begin{proof}
If $x \in (A \cup B)^c$, then $x$ is not in $A$ or $B$, so by the propositional DeMorgan's rule, $x \in A^c \cap B^c$.
Similarly, if $x \in A^c \cap B^c$, $x \not\in A$ and $x \not\in B$, so by DeMorgan's rule, $x \not\in A \cap B$, thus, $x \in (A \cap B)^c$.
\end{proof}

\subsubsection*{Exercise 1.2.6-a}

\begin{proof}
Suppose $a, b \geq 0$. Then $|a + b| = a + b = |a| + |b|$ so $|a + b| \leq |a| + |b|$.
If $a, b \leq 0$, then we can rewrite $a = -x$ and $b = -y$ where $x, y \geq 0$, so $|- x - y| = |-1||x + y| = x + y = |-x| + |-y|$,
so $|a + b| \leq |a| + |b|$.
\end{proof}

\subsubsection*{Exercise 1.2.6-b}

\begin{proof}
If $a, b \geq 0$, $|a| = a$, $|b| = b$, so $(a + b)^2 = (|a| + |b|)^2$.

If $a, b \leq 0$, we can factor out the -1, so $(-a -b)^2 = (a + b)^2$ and since $|a| = -a$ and $|b| = -b$, we get $(a + b)^2 = (|a| + |b|)^2$.

Finally suppose that $a$ and $b$ have different signs. Then, without loss of generality, suppose $a \geq 0$ and $b \leq 0$, so that $b = -c$ for $c \geq 0$.
Then $(a + b)^2 = a^2 -2bc + c^2$ while $(|a| + |b|)^2 = a^2 + 2ac + c^2$, and since $a, c \geq 0$, we know that $(a + b)^2 \leq (|a| + |b|)^2$.

$0 \leq a \leq b$ if and only if $0 \leq a^2 \leq b^2$, so because $|a + b| \geq 0$ and $|a| + |b| \geq 0$,
$|a + b| \leq |a| + |b|$ if and only if $|a + b|^2 \leq (|a| + |b|)^2$.
Then, since $(a + b)^2 \leq (|a| + |b|)^2$ and $|a + b|^2 = (a + b)^2$ (because if $a + b < 0$, the -1 multiplies out), we know that $|a + b| \leq |a| + |b|$.

\end{proof}

\subsubsection*{Exercise 1.2.6-c}

\begin{proof}
$a - b = a - c + c - d + d - b$, applying $|\cdot|$ to both sides, $|a - b| \leq |a - c| + |c - d + d - b|$ by the triangle inequality, and the triangle
inequality gives $|c - b| \leq |c - d| + |d - b|$, so $|a - b| \leq |a - c + c - d + d - b|$.
\end{proof}

\subsubsection*{Exercise 1.2.6-d}

\begin{proof}
$||a| - |b|| = ||a - b + b| - |b|| \leq ||a - b| + |b| - |b|| = ||a - b|| = |a - b|$, where the inequality is fron the triangle inequality.
\end{proof}

\subsubsection*{Exercise 1.2.7-a}

$f(A) = [0, 4]$ and $f(B) = [1, 16]$. $f(A \cap B) = [1, 4] = f(A) \cap f(B)$. $f(A \cup B) = [0, 16] = [0, 4] \cup [1, 16]$.

\subsubsection*{Exercise 1.2.7-b}

$A = [-1, 0]$ and $B = [0, 1]$. Then while $f(A \cap B) = \{0\}$ but $f(A) \cap f(B) = [0, 1]$.

\subsubsection*{Exercise 1.2.7-c}

\begin{proof}
If $x \in g(A \cap B)$, then for some $y \in A$ and $y \in B$, $y = g(x)$, so $x \in g(A)$ and $x \in g(B)$, so $x \in g(A) \cap g(B)$.
\end{proof}

\subsubsection*{Exercise 1.2.7-d}

\begin{lemma}
Let $g: \R \to \R$ and $A, B \subset \R$. Then, $g(A \cup B) = g(A) \cup g(B)$.
\end{lemma}

\begin{proof}
If $x \in g(A \cup B)$, then for some $y \in A$ or $y \in B$, $y = g(x)$, so $x \in g(A)$ or $x \in g(B)$, so $x \in g(A) \cup g(B)$.

In reverse, without loss of generality, suppose $x \in g(A)$, so there is a $y \in A$ so that $g(y) = x$, and so $y \in A \cup B$, putting $x \in g(A \cup B)$.
\end{proof}

\subsubsection*{Exercise 1.2.8-a}

Let $f(x) = x + 1$. Then if $f(x) = f(y)$ for $x, y \in \N$, $x + 1 = y + 1$, so $x = y$; meaning $f$ is 1-1.
Furthermore, there is no $x \in \N$ so that $x + 1 = 0$, so $f$ is not onto.

\subsubsection*{Exercise 1.2.8-b}

Let $f(x) = \begin{cases} {x \over 2} & 2 \mid x \\ {x - 1 \over 2} & 2 \nmid x\end{cases}$, then $f(2y) = y$ for $y \in \N$ so $f$ is onto but $f(0) = f(1) = 0$ so $f$ is not 1-1.

\subsubsection*{Exercise 1.2.8-b}

Put $f(x) = \begin{cases} {x \over 2} & 2 \mid x \\ - {x + 1 \over 2} & 2 \nmid x\end{cases}$. $f$ is onto since if $y \in \Z$, $y \geq 0$, $f(2y) = y$ and if $y < 0$, $f(-2y - 1) = y$.
Let $x, y \in \N$ such that $f(x) = f(y)$. Then $2 \mid x$ if and only if $2 \mid y$. If $2 \mid x$, ${x \over 2} = {y \over 2}$ so that $x = y$.
Similarly, if $2 \nmid x$, $- {x + 1 \over 2} = - {y + 1 \over 2}$, so $x + 1 = y + 1$ meaning $x = y$. Hence $f$ is 1-1.

\subsubsection*{Exercise 1.2.9-a}

$f^{-1}(A) = [-2, 2]$ and $f^{-1}(B) = [-1, 1]$.

$f^{-1}(A \cap B) = [-1, 1] = [-2, 2] \cap [1, -1] = f^{-1}(A) \cap f^{-1}(B)$.

$f^{-1}(A \cup B) = [-2, 2] = [-2, 2] \cup [1, -1] = f^{-1}(A) \cup f^{-1}(B)$.

\subsubsection*{Exercise 1.2.9-b}

Let $g: \R \to \R$ and $A, B \subset \R$.

\begin{lemma}
$g^{-1}(A \cap B) = g^{-1}(A) \cap g^{-1}(B)$.
\end{lemma}

\begin{proof}
$x \in g^{-1}(A \cap B)$ if and only if $f(x) \in A \cap B$ meaning $x \in f^{-1}(A)$ and $x \in f^{-1}(B)$ so $x \in g^{-1}(A) \cap g^{-1}(B)$ by definition.
\end{proof}

\begin{lemma}
$g^{-1}(A \cup B) = g^{-1}(A) \cup g^{-1}(B)$.
\end{lemma}

\begin{proof}
$x \in g^{-1}(A \cup B)$ if and only if $f(x) \in A \cup B$ meaning $x \in f^{-1}(A)$ or $x \in f^{-1}(B)$ so $x \in g^{-1}(A) \cup g^{-1}(B)$ by definition.
\end{proof}

\subsubsection*{Exercise 1.2.10-a}

This is false in the reverse direction. While if $a < b$, for all $\e > 0$, $a < b + \e$, if $a = b$, for all $\e > 0$, we have $a < b + \e$.

\subsubsection*{Exercise 1.2.10-b}

Similar to the above, if $a = b$, we still have that $a < b + \e$ for all $\e > 0$.

\subsubsection*{Exercise 1.2.10-c}

Suppose $a \leq b$ and $\e > 0$. Then since $a - b \leq 0$, $a - b < \e$, so $a < b + \e$.
Conversely, suppose $a > b$. Then $a - b > 0$, and we can take $\e = {a - b \over 2} > 0$ and find that $b + \e = {a + b \over 2} < a$.

\end{document}
