\documentclass{article}
\usepackage[utf8]{inputenc}
\usepackage{amsmath, amssymb, amsthm}
\usepackage{tikz}
\usepackage{wrapfig}
\usepackage[margin=15mm]{geometry}


\newcommand{\R}{\mathbb{R}}
\newcommand{\Z}{\mathbb{Z}}
\newcommand{\N}{\mathbb{N}}
\newcommand{\Q}{\mathbb{Q}}
\newcommand{\diag}{\text{diag}}
\newcommand{\rad}{\text{rad\ }}

\newtheorem*{lemma}{Lemma}

\begin{document}
\begin{center}
\subsection*{Abbot Chapter 1 Section 1}
\end{center}

Heman Gandhi
\hfill
2024-10-22\\

\subsubsection*{Exercise 1.2.1-a}

\textit{This follows the $exercise\_1\_2\_1$ section in Lean, so might be overly detailed. I'm also pretentiously going to use lemmas that correspond to the Lean theorems.}

\begin{lemma}
First, we show that for $n \in \N$, we know that there is an $x \in \N$ so that $3x = n$,
$3x + 1 = n$, or $3x + 2 = n$.
\end{lemma}

\begin{proof}
We proceed by induction, showing that $3 \cdot 0 = 0$, then
if the statement holds for $n$, we proceed by cases and show that for the same $x$ we reached for
$n$, we either have $3x + 1 = n + 1$, $3x + 2 = n + 1$, or $3(x + 1) = n + 1$.
\end{proof}

\begin{lemma}
If $n \in \N$, then $3 \nmid n$ if and only if for some $x \in \N$, $3x + 1 = n$
or $3x + 2 = n$.
\end{lemma}

\begin{proof}
We can prove the forward direction by the above statement: since $3x = n$ would contradict
the assumption that $3 \nmid n$.

The reverse direction is simplest by contradiction: if we have the $x$ with
remainder 1 or 2, we cannot find some $y$ so that $3y = n$, since we'd for the equation $3(y - x) = r$ for $r$
being 1 or 2, which is absurd since 3 cannot divide a non-zero number less than itself.
\end{proof}

For the final lemma: we show that

\begin{lemma}
For $a, b \in \N$ if $3 \mid ab$, $3 \mid a$ or $3 \mid b$.
\end{lemma}

\begin{proof}
We show the contrapositive: assuming $3 \nmid a$ and $3 \nmid b$, we have $3x_a + r_a = a$ and
$3x_b + r_b = b$ for $x_a, x_b \in \N$ and $r_a, r_b \in \{1, 2\}$ from the forward direction of the above.
We compute $ab$ with the above in all four cases:
\begin{enumerate}
    \item if $r_a, r_b = 1$, then $ab = 9 x_a x_b + 3 x_a + 3 x_b + 1 = 3(3 x_a x_b + x_a + x_b) + 1$;
    \item if $r_a = 1, r_b = 2$, then $ab = 9 x_a x_b + 6 x_a + 3 x_b + 1 = 3(3 x_a x_b + 2 x_a + x_b) + 2$;
    \item if $r_a = 2, r_b = 1$, then $ab = 9 x_a x_b + 3 x_a + 6 x_b + 1 = 3(3 x_a x_b + x_a + 2 x_b) + 2$;
    \item if $r_a, r_b = 2$, then $ab = 9 x_a x_b + 6 x_a + 6 x_b + 4 = 3(3 x_a x_b + 2 x_a + 2 x_b + 1) + 1$.
\end{enumerate}
In all the cases, we can express $ab = 3y + r$ for $y \in \N$ and $r \in \{1, 2\}$ and apply the backwards direction of
the lemma above to conclude $3 \nmid ab$, showing the contrapositive.
\end{proof}

For contradiction, let $a, b \in \N$ and ${a^2 \over b^2} = 3$. Without loss of generality, we can assume that $a$ and $b$ don't
share factors. Rewriting this as $a^2 = 3 b^2$, we see that $3 \mid a^2$, so the above gives us that $3 \mid a$.
Hence, we write $a = 3d$, so $a^2 = 9 d^2 = 3 b^2$, which means that $b^2 = 3 d^2$, so $3 \mid b$. This contradicts the assumption
that $a$ and $b$ don't share factors.

\end{document}
